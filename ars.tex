\documentclass[a4paper,10pt]{extarticle}
\usepackage{amssymb,amsmath,amsthm,amsfonts}
\usepackage{multicol,multirow}
\usepackage{calc}
\usepackage{ifthen}
\usepackage{tabularx}
\usepackage{listings}
\usepackage{caption}
\usepackage[utf8]{inputenc}
\usepackage[landscape]{geometry}
\usepackage[colorlinks=true,citecolor=blue,linkcolor=blue]{hyperref}
\usepackage{accents}
\newcommand{\vect}[1]{\accentset{\rightharpoonup}{#1}}


\DeclareCaptionLabelFormat{algocaption}{Algorithm \thenalg} % defines a new caption label as Algorithm x.y

\lstnewenvironment{algorithm}[1][] %defines the algorithm listing environment
{   
    %\refstepcounter{nalg} %increments algorithm number
    %\captionsetup{labelformat=algocaption,labelsep=colon} %defines the caption setup for: it ises label format as the declared caption label above and makes label and caption text to be separated by a ':'
    \lstset{ %this is the stype
        mathescape=true,
        %frame=tB,
        %numbers=left, 
        numberstyle=\tiny,
        basicstyle=\scriptsize, 
        keywordstyle=\color{black}\bfseries\em,
        keywords={,vhod, izhod, vrni, dokler, izvajaj, konec, zanke} %add the keywords you want, or load a language as Rubens explains in his comment above.
        %numbers=left,
        %xleftmargin=.04\textwidth,
        %#1 % this is to add specific settings to an usage of this environment (for instnce, the caption and referable label)
    }
}
{}


\ifthenelse{\lengthtest { \paperwidth = 11in}}
    { \geometry{top=.5in,left=.5in,right=.5in,bottom=.5in} }
	{\ifthenelse{ \lengthtest{ \paperwidth = 297mm}}
		{\geometry{top=1cm,left=1cm,right=1cm,bottom=1cm} }
		{\geometry{top=1cm,left=1cm,right=1cm,bottom=1cm} }
	}
\pagestyle{empty}
\makeatletter
\renewcommand{\section}{\@startsection{section}{1}{0mm}%
                                {-1ex plus -.5ex minus -.2ex}%
                                {0.5ex plus .2ex}%x
                                {\normalfont\large\bfseries}}
\renewcommand{\subsection}{\@startsection{subsection}{2}{0mm}%
                                {-1explus -.5ex minus -.2ex}%
                                {0.5ex plus .2ex}%
                                {\normalfont\normalsize\bfseries}}
\renewcommand{\subsubsection}{\@startsection{subsubsection}{3}{0mm}%
                                {-1ex plus -.5ex minus -.2ex}%
                                {1ex plus .2ex}%
                                {\normalfont\small\bfseries}}
\makeatother
\setcounter{secnumdepth}{0}
\setlength{\parindent}{0pt}
\setlength{\parskip}{0pt plus 0.5ex}
% -----------------------------------------------------------------------

\title{Diskretne strukture}

\begin{document}

\raggedright
\footnotesize
\begin{multicols}{3}
\section{Predstavitev števil}
$r\ \dots $ baza\\
$b = (b_n-1, ..., b_0)\ \dots $ število v bazi $r$ s števkami $b_1$\\ 
\begin{itemize}
    \item \textbf{nepredznačeno celo število}
    $V = \sum_{i=0}^{n}b_ir^i$
    \item \textbf{predznak in velikost (PV)} 
    $V = (-1)^{b_{n-1}} \sum_{i=0}^{n-2} b_i2^i$ \\
    prvi bit ($b_{n-1}$) predstavlja predznak ($1\to -$,$0 \to +$), ostali pa velikost.
    \item \textbf{z odmikom}
    $V = \sum_{i=0}^{n-1} \ -\ \underbrace{2^{n-1}}_\textmd{odmik}$ \\
    \item \textbf{eniški komplement (1'K)}
    $V = \sum_{i=0}^{n-1} b_i 2^i - b_{n-1}(2^n-1) $ \\
    prvi bit predstavlja predznak ($1\to -$,$0 \to +$). Negativno število dobimo tako, da invertiramo vse bite pozitivnega števila.
    \item \textbf{dvojiši komplement (2'K)}
    $V = \sum_{i=0}^{n-1} b_i 2^i - b_{n-1}2^n$ \\
    Negativno število dobimo tako da bite inveritramo in prištejemo 1.
    \item \textbf{pozicijska notacija}
    $V = \sum_{i=-m}^{n-1}b_ir^i$
\end{itemize}
\subsection{Plavajoča vejica}
\begin{itemize}
    \item enojna natančnost \emph{32 bit}\\
    prvi bit je predznak $S$ ($1\to -$,$0 \to +$)\\
    sledi $8$-biten eksponent $e$ zapisan z odmikom ($e = E-127$)\\
    sledi $23$-bitna mantisa $m$\\
    prvi bit mantise je implicitno podan: če je eksponent $E = 0$ (v tem primeru je $e = -126$ in ne $-127$) je mantisa oblike $0,m$, sicer je oblike $1,m$.\\
    če so vsi biti eksponenta $1$ in vsi biti mantise $0$, je vrednost $\pm\infty$.\\
    če so vsi biti eksponenta $1$ in vsi biti mantise niso $0$, je vrednost NaN.\\
    \item dvojna natančnost \emph{64 bit}\\
    prvi bit je predznak\\
    sledi $11$-biten eksponent z odmikom $1023$\\
    in $53$-bitna mantisa\\
    \item normirana vrednost $(-1)^S\cdot(1,m)\cdot 2^e$
    \item denormirana vrednost $(-1)^S\cdot(0,m)\cdot 2^{e+1}$
    \item aritmetika v plavajoči vejici\\
    Zaokrožujemo k najbližji vrednosti, ki se jo da predstaviti (preferiramo soda števila).\\
    Pri računanju mantiso podaljšamo za 3 bite (\emph{varovalni, zaokroževalni, lepljivi})
\end{itemize}
\subsubsection{Seštevanje v plavajoči vejici}
\begin{itemize}
    \item Obe števili zapišemo z večjim eksponentom (premik mantise, če izpadajo kake enice, se shranijo v leplijvem bitu)
    \item Mantisi seštejemo, če se pojavi prenos naprej, zmanjšamo mantiso in povečamo eksponent.
\end{itemize}
\subsubsection{Množenje v plavajoči vejici}
\begin{itemize}
    \item eksponenta seštejemo
    \item mantisi zmnožimo v fiksni vejici
\end{itemize}


\textbf{Prenos} se pojavi, ko rezultat neke operacije preseže obseg števil. (nanaše se le na operacije z \emph{nepredznačenimi} števili; pri 2'K se ignorira)\\
\textbf{Preliv} se pojavi če ima rezultat drugačen predznak kot števili.

\section{CPE}
\begin{equation*}
    \begin{aligned}    
        t_{CPE}\ \dots\ & \textmd{dolžina med dvema periodama} \\
        f_{CPE} = 1/t_{CPE}\ \dots\ & \textmd{frekvenca ure} \\
        CPI = \sum_{i=0}^{n} CPI_i p_i \ \dots\ & \textmd{povprečno število urinih preiod na ukaz} \\  
        MIPS = 1/(CPI\cdot t_{CPE}\cdot 10^6)\ \dots\ & \textmd{million instructions per second} \\  
    \end{aligned}
\end{equation*}

\section{Predpomnilnik}
\begin{equation*}
    \begin{aligned}    
        t_{ap}\ \dots\ & \textmd{čas dostopa do predpomnilnika} \\
        t_{ag}\ \dots\ & \textmd{čas dostopa do glavnega pomnilnika} \\
        H\ \dots\ & \textmd{verjetnost zadetka}\\
        t_B\ \dots\ & \textmd{čas za prenos celega bloka}\\
        t_a = t_{ap} + (1-H)t_B\ \dots\ & \textmd{povprečen čas dostopa}\\
    \end{aligned}
\end{equation*}

\subsubsection{Set asociativni predpomnilnik (SAPP)}
\begin{equation*}
    \begin{aligned}    
        n\ \dots\ & \textmd{dolžina naslova} \\
        B = 2^b\ \dots\ & \textmd{Število besed v bloku} \\
        S = 2^s\ \dots\ & \textmd{število setov, vsak set je majhen APP} \\
        E = 2^e\ \dots\ & \textmd{število blokov v setu (\textbf{stopnja asociativnosti})} \\
        M_b = S\cdot E\ \dots\ & \textmd{število blokov v PP}\\
        M = S\cdot E\cdot B\ \dots\ & \textmd{velikost PP}\\
    \end{aligned}
\end{equation*}
\begin{itemize}
    \item vsak naslov ($A_i$) iz GP se lahko preslika le v en set ($S_i$).
    \[S_i = \underbrace{A_i[n-1 : b]}_\textmd{zgornji biti naslova} \% 2^s\ \dots\ \textmd{naslov seta }\textit{(predpomnilniški indeks)} \]
    \item iz naslova ($A_i$) se prebere naslov bloka (zgornjih $n-b$ bitov) in naslov besede (spodnjih $b$ bitov).
\end{itemize}

\subsubsection{Čisti asociativni predpomnilnik (APP)}
\begin{itemize}
    \item samo en set $S=1$, posledično $E=M_b$
    \item vsak blok sprejme katerikoli del iz GP
    \item največja verjetnost zadetka
\end{itemize}

\subsubsection{Direktni predpomnilnik}
To je SAPP kjer je v vsakem setu le en blok $E=1$.

\subsection{Zgrešitve}
Ob zgrešitvi se v PP prenese cel nov blok. Če v setu ni več prostora, se en blok zamenja (naključna strategija, \emph{Least Recently Used} strategija).
\subsubsection{Vrste zgrešitev}
\begin{itemize}
    \item \textbf{obvezne zgrešitve} pojavijo se vsakič, ko nek blok pomnilnika zahtevamo \emph{prvič}.
    \item \textbf{velikostne zgrešitve} ko moramo naložiti nek blok, ki smo ga sicer že imeli a smo ga morali zamenjati zaradi prostorske stiske.
    \item \textbf{konfliktne zgrešitve} ko moramo naložiti nek blok, ki smo ga sicer že imeli a smo ga morali zamenjati, ker se je nek drug blok preslikal v isti set.
\end{itemize}

\subsection{Vpliv predpomnilnika na hitrost}
\[CPE_\textmd{čas} = (CPE_{\textmd{periode}_\textmd{izvrš.}} + CPE_{\textmd{periode}_\textmd{čaka.}})t_{CPE}\]
\[CPE_{\textmd{periode}_\textmd{izvrš.}} = I\cdot CPI_\textmd{idealni}\]
\[CPE_{\textmd{periode}_\textmd{čaka.}} = N(1-H)K_Z\]
$I\ \dots\ $št. ukazov\\
$N\ \dots\ $št. pomnilniških dostopov\\
$H\ \dots\ $povprečna verjetnost zadetka PP\\
$K_Z\ \dots\ $povprečna zgrešitvena kazen\\
$CPI_\textmd{idealni}\ \dots\ $št. period na ukaz s predpostavko, da ni zgrešitev\\

\section{Sklad}
\begin{tabular}{ r|l } 
    r0 & ničla \\ 
    r1-r23 & splošno namenski registri \\ 
    r24 & prvi parameter ob klicu podprograma \\
    r25 & drugi parameter ob klicu podprograma \\
    r26 & bazni register za dolge skoke \\
    r27 & bazni register za dolge klice \\
    r28 & return value \\
    r29 & \textbf{FP} frame pointer \\
    r30 & \textbf{SP} stack pointer \\
    r31 & return address \\ 
\end{tabular}
\\\
\\\
\\\
\begin{tabular}{ r | r}
\begin{lstlisting}
push reg:
    sw 0(r30), reg
    subui r30, r30, #4
\end{lstlisting}
&
\begin{lstlisting}
pop reg:
    addui r30, r30, #4
    lw reg, 0(r30)
\end{lstlisting}
\end{tabular}

\subsection{Klic podprograma}
\begin{itemize}
    \item prva dva parametra shranimo v r24 in r25
    \item ostale parametre porinemo na sklad
    \item pokličemo podprogram z ukazom 
\end{itemize}
\subsubsection{Klicani podprogram ob vstopu}
\begin{itemize}
    \item na sklad porine \textbf{povratni naslov} (push r31)
    \item sa sklad porine \textbf{star kazalec na okvir} (push r29)
    \item \textbf{kazalec na okvir} nastavi na \textbf{skladovni kazalec} (r29 $\leftarrow$ r30)
    \item po potrebi rezervira prostor na skladu za lokalne spremenljivke (skladovni kazalec premakne/pomanjša za velikost lokalnih spremenljivk)
    \item na sklad shrani vse registre, ki jih bo spreminjal
\end{itemize}
\subsubsection{Klicani program med izvajanjem}
\begin{itemize}
    \item parametrov in lokalnih spremenljivk dostopamo prek kazalca na okvir
    \begin{itemize}
        \item prva lokalna spremenljivka je na MEM[FP]
        \item star kazalec na okvir je na MEM[FP+4]
        \item povratni naslov je na MEM[FP+8]
        \item zadnji parameter je na MEM[FP+12]
    \end{itemize}
\end{itemize}
\subsubsection{Klicani podprogram pred izstopom}
\begin{itemize}
    \item v r28 shrani vrednost, ki jo vrača
    \item s sklada obnovi vse shranjene registre
    \item s sklada pobriše vse lokalne spremenljivke (r30 $\leftarrow$ r29)
    \item s sklada obnovi staro vrednost kazalca na okvir (pop r29)
    \item s sklada obnovi povratni naslov (pop r31)
    \item skoči na povratni naslov
\end{itemize}
\subsubsection{Po vrnitvi iz podprograma}
\begin{itemize}
    \item izbrišemo parametre iz sklada
\end{itemize}

\section{Cevovod}
Procesor HIP deluje v 5 stopnjah (ki se odvijajo naenkrat)
\begin{itemize}
    \item \textbf{IF} instruction fetch - \emph{prevzem instrukcije iz GP}
    \item \textbf{ID} instruction decode - \emph{dekodiranje ukaza in branje iz registrov}
    \item \textbf{EX} execute - \emph{izvršitev operacije (na ALU)}
    \item \textbf{MEM} memory - \emph{dostop do pomnilnika}
    \item \textbf{WB} write back - \emph{shranjevanje reultata (v reg)}
\end{itemize}
\section{Cevovodne nevarnosti}
\subsubsection{Strukturne nevarnosti}
Več stopenj uporablja isto enoto. \emph{ne pri HIP}
\subsubsection{Podatkovne nevarnosti}
Ukaz kot parameter potrebuje še neizračunan rezultat prejšnjega ukaza \emph{samo v ID}. Rešujemo jih lahko:
\begin{itemize}
    \item \textbf{programsko} - vstavljanje nop za problematične ukaze
    \item \textbf{cevovodna zaklenitev} - ko procesor (v stopnji ID) zazna nevarnost, v cevovod vstavlja "mehurčke"
    \item \textbf{premoščanje} - rezultate iz EX, MEM ali WB prenesemo v ID, če to ni mogoče (pri load), vstavimo mehurček
    \item \textbf{razvrščanje} - ukaze razvrstimo tako, da odpravimo nevarnost
\end{itemize}
\subsubsection{Kontrolne nevarnosti}
Pri ukazih (skokih), ki spreminjajo PC.\\
\emph{HIP nov naslov izračuna že v ID, pri pogojnih skokih se PC lahko spremeni v EX (vsebina IF in ID postane neveljavna)} Procesor predpostavi, da skoka ne bo.\\
Rešitve:
\begin{itemize}
    \item \textbf{vstavljanje mehurčkov} v primeru skoka se v stopnji IF in ID vstavita mehurčka, sicer pa gre vse normalno naprej.
    \item \textmd{predikcija skočnega pogoja}
    \begin{itemize}
        \item \textbf{statična} med izvrševanjem se ne spreminja
        \item \textbf{dinamična} procesor si zapomni vrednost prejšnjih skokov
    \end{itemize}
    \item \textbf{zakasnjeni skoki} : v \emph{skočne reže} (pri HIP 2 ukaza za skokom) damo ukaze, ki bi se morali izvesti pred skokom in ne vplivajo na pogoj skoka.
\end{itemize}


\end{multicols}
\end{document}
